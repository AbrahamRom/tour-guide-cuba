\documentclass{article}
\usepackage[spanish]{babel}
\usepackage{geometry}
\usepackage{amsmath}
\usepackage{amssymb}
\usepackage{booktabs}
\usepackage{graphicx}
\usepackage{tikz}
\usetikzlibrary{shapes, arrows, positioning}

\title{Planificador de Hospedaje en Hoteles}
\author{Sistema de Optimización de Itinerarios}
\date{\today}

\begin{document}

\maketitle

\section{Introducción}
Este sistema utiliza metaheurísticas para optimizar itinerarios de hoteles considerando:
\begin{itemize}
    \item Presupuesto total
    \item Calidad (estrellas) de hoteles
    \item Preferencia por minimizar cambios de hotel
    \item Restricciones de destino y disponibilidad
\end{itemize}

\section{Función Objetivo (Fitness)}
\label{sec:fitness}
Definida en \texttt{fitness.py}, evalúa soluciones mediante:

\begin{equation}
\text{fitness} = \alpha \times \text{stars\_norm} + \beta \times (1 - \text{cost\_norm}) + \gamma \times (1 - \text{changes\_norm})
\end{equation}

Donde:
\begin{align*}
\text{stars\_norm} &= \frac{\sum \text{estrellas}}{\text{noches} \times \text{max\_stars}} \\
\text{cost\_norm} &= \min\left(\frac{\text{costo\_total}}{\text{presupuesto}}, 1\right) \\
\text{changes\_norm} &= \frac{\text{cambios\_hotel}}{\text{noches} - 1}
\end{align*}

\section{Metaheurísticas Implementadas}

\subsection{Búsqueda en Profundidad (DFS)}
\label{subsec:dfs}
Implementada en \texttt{graph\_explorer.py} y \texttt{graph\_node.py}.

\begin{itemize}
    \item \textbf{Estrategia}: Búsqueda exhaustiva con backtracking
    \item \textbf{Representación}: Árbol donde cada nodo contiene:
    \begin{verbatim}
class GraphNode:
    night: int
    hotel: Hotel
    budget_left: float
    stars_accum: int
    path: List[Tuple[int, Hotel]]
    \end{verbatim}
    
    \item \textbf{Flujo}:
    \begin{enumerate}
        \item Inicializa con todos los hoteles válidos para la primera noche
        \item Expande nodos generando hijos válidos (presupuesto)
        \item Ordena opciones por: estrellas $\downarrow$ y relación estrellas/precio $\downarrow$
        \item Mantiene la mejor solución completa
    \end{enumerate}
    
    \item \textbf{Ventaja}: Óptimo garantizado
    \item \textbf{Limitación}: Complejidad exponencial $O(h^n)$
\end{itemize}

\subsection{Optimización por Colonia de Hormigas (ACO)}
\label{subsec:aco}
Implementada en \texttt{aco\_planner.py}.

\begin{figure}[h]
\centering
\begin{tikzpicture}[node distance=1.5cm, auto]
    \tikzstyle{block} = [rectangle, draw, text width=12em, text centered, rounded corners, minimum height=4em]
    \tikzstyle{line} = [draw, -latex']
    
    \node [block] (init) {Inicializar matriz de feromonas};
    \node [block, below of=init] (construct) {Construir soluciones para $k$ hormigas};
    \node [block, below of=construct] (evaluate) {Evaluar fitness de cada solución};
    \node [block, below of=evaluate] (update) {Actualizar feromonas};
    \node [block, below of=update] (best) {Guardar mejor solución global};
    \node [block, left of=construct, node distance=4cm] (stop) {Devolver mejor solución};
    
    \path [line] (init) -- (construct);
    \path [line] (construct) -- (evaluate);
    \path [line] (evaluate) -- (update);
    \path [line] (update) -- (best);
    \path [line] (best) -- ++(0,-1) -- ++(-4,0) |- (construct);
    \path [line] (best) -| (stop);
    
    \node at (5,-8) [text width=8em] {Itera $n$ veces};
\end{tikzpicture}
\caption{Diagrama de flujo del algoritmo ACO}
\end{figure}

\textbf{Ecuación de probabilidad}:
\begin{equation}
P(i) = \frac{\tau_i \times \eta_i}{\sum_{j} \tau_j \times \eta_j}
\end{equation}
Donde:
\begin{itemize}
    \item $\tau_i$: Feromona en hotel $i$
    \item $\eta_i$: Heurística $\frac{\text{estrellas}_i}{\text{precio}_i}$
    \item Penalización por cambio: $\eta_i \times 0.8$ si $i \neq$ hotel anterior
\end{itemize}

\textbf{Actualización de feromonas}:
\begin{align*}
\tau_i^{(t+1)} &= (1 - \rho) \times \tau_i^{(t)} + \Delta\tau_i \\
\Delta\tau_i &= \sum \text{fitness}(\text{soluciones que usan } i)
\end{align*}

\subsection{Optimización por Enjambre de Partículas (PSO)}
\label{subsec:pso}
Implementada en \texttt{pso\_planner.py}.

\begin{figure}[h]
\centering
\begin{tikzpicture}[node distance=1.5cm, auto]
    \tikzstyle{block} = [rectangle, draw, text width=10em, text centered, rounded corners, minimum height=4em]
    \tikzstyle{line} = [draw, -latex']
    
    \node [block] (init) {Inicializar partículas y velocidades};
    \node [block, below of=init] (velocity) {Actualizar velocidades};
    \node [block, below of=velocity] (position) {Actualizar posiciones};
    \node [block, below of=position] (repair) {Reparar soluciones inválidas};
    \node [block, below of=repair] (evaluate) {Evaluar fitness};
    \node [block, below of=evaluate] (update) {Actualizar mejores locales/globales};
    \node [block, right of=update, node distance=4cm] (output) {Devolver mejor solución};
    
    \path [line] (init) -- (velocity);
    \path [line] (velocity) -- (position);
    \path [line] (position) -- (repair);
    \path [line] (repair) -- (evaluate);
    \path [line] (evaluate) -- (update);
    \path [line] (update) -- ++(0,-1) -- ++(-4,0) |- (velocity);
    \path [line] (update) -- (output);
    
    \node at (5,-9) [text width=8em] {Itera $n$ veces};
\end{tikzpicture}
\caption{Diagrama de flujo del algoritmo PSO}
\end{figure}

\textbf{Ecuaciones de actualización}:
\begin{align*}
v_{id}^{(t+1)} &= w \cdot v_{id}^{(t)} + c_1 r_1 (p_{id} - x_{id}^{(t)}) + c_2 r_2 (g_d - x_{id}^{(t)}) \\
P(\text{cambio}) &= \frac{1}{1 + e^{-v_{id}^{(t+1)}}}
\end{align*}

\textbf{Mecanismos clave}:
\begin{itemize}
    \item Representación: Vector de índices de hoteles
    \item Reparación: Reemplazar hoteles que exceden presupuesto
    \item Sigmoide: Transforma velocidad a probabilidad de cambio
\end{itemize}

\section{Comparativo de Algoritmos}
\begin{table}[h]
\centering
\begin{tabular}{lccc}
\toprule
Característica & DFS & ACO & PSO \\
\midrule
Optimalidad & Garantizada & Heurística & Heurística \\
Complejidad temporal & $O(h^n)$ & $O(n \cdot k \cdot h)$ & $O(n \cdot p \cdot d)$ \\
Espacio de búsqueda & Pequeños & Grandes & Grandes \\
Paralelización & Difícil & Alta & Muy Alta \\
Noches recomendadas & $<7$ & $<15$ & $>10$ \\
Sensibilidad a parámetros & Baja & Media & Alta \\
\bottomrule
\end{tabular}
\caption{Comparativo de metaheurísticas}
\end{table}

\section{Conclusiones}
\begin{itemize}
    \item \textbf{DFS}: Ideal para problemas pequeños con garantía de optimalidad
    \item \textbf{ACO}: Balance óptimo entre exploración y explotación
    \item \textbf{PSO}: Alto rendimiento en problemas grandes con paralelización
    \item \textbf{Función fitness}: Combina calidad, costo y estabilidad en métrica unificada
\end{itemize}

\end{document}